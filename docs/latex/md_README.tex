\href{https://travis-ci.org/RajPShinde/Robot_Controller_Module}{\tt } \href{https://coveralls.io/github/RajPShinde/Robot_Controller_Module?branch=master}{\tt } \href{https://github.com/RajPShinde/Robot_Controller_Module/blob/master/LICENSE}{\tt } \subsection*{\href{https://github.com/RajPShinde/Robot_Controller_Module/tree/master/docs}{\tt } }

\subsection*{Authors}


\begin{DoxyItemize}
\item {\bfseries Raj Prakash Shinde} -\/ {\itshape Sprint 1-\/ Driver \& Sprint 2-\/ Navigator} -\/ \href{https://github.com/RajPShinde}{\tt Git\+Hub} ~\newline
I am a Masters in Robotics Engineering student at the University of Maryland, College Park. My primary area of interest are Legged Robotics and Automation.
\item {\bfseries Prasheel Renkuntla} -\/ {\itshape Sprint 1-\/ Navigator \& Sprint 2-\/ Driver} -\/ \href{https://github.com/Prasheel24}{\tt Git\+Hub} ~\newline
I am a Master\textquotesingle{}s in Robotics Engineering student at the University of Maryland, College Park. My primary area of interest is in Vision integrated Robot Systems.
\end{DoxyItemize}

\subsection*{Overview}

This is a controller module for a robot that uses an Ackermann Steering Model, This controller is to be implemented in a four wheeled Robot made by A\+C\+ME Robotics.

\paragraph*{Description}

A controller is build for a 4 wheeled robot with an Ackermann steering to navigate through its environment. The controller consist of a P\+ID algorithm which ensures that the velocity converges to the set point, and a Steer Algorithm that helps the robot turn. ~\newline
The P\+ID Algorithm is a control loop mechanism that calculates the error and applies correction through proportional, integral and derivative gains. The Steer Algorithm is developed to turn the robot, which is done by calculating the length of an arc inscribed between the current robot heading and target heading, the length when divided by the robot velocity gives the time for which the wheels need to be kept at angles given by Ackermann steering Model. ~\newline
The input to the controller will be provided by the perception model developed by the A\+C\+ME Robotics. ~\newline
The Demonstration of the controller will be given by plotting a graph that shows convergence of velocity \& Heading angle to the targets with respect to time.

\paragraph*{Features}


\begin{DoxyItemize}
\item Velocity control during turning.
\item Protection against Skidding, by limiting velocity during turning.
\item Protection agains over volting the Motors.
\end{DoxyItemize}

\paragraph*{Application}


\begin{DoxyItemize}
\item Mobile Wheeled Robots
\end{DoxyItemize}

\subsection*{Sprint Planning and Discussion}

Sprint-\/ \href{https://docs.google.com/document/d/1w6U49tyKj9MFhaVziZ5MEGcaylmxSBOyGzClIz9lFA8/edit?usp=sharing}{\tt https\+://docs.\+google.\+com/document/d/1w6\+U49ty\+Kj9\+M\+Fha\+Vzi\+Z5\+M\+E\+Gcaylmx\+S\+B\+Oy\+Gz\+Cl\+Iz9l\+F\+A8/edit?usp=sharing}

\subsection*{Agile Iterative Process Log}

Log-\/ \href{https://docs.google.com/spreadsheets/d/1LFQMKbuGeusgmI7IMbjiw-RJrt9jNgej0F8SvvfyJjY/edit?usp=sharing}{\tt https\+://docs.\+google.\+com/spreadsheets/d/1\+L\+F\+Q\+M\+Kbu\+Geusgm\+I7\+I\+Mbjiw-\/\+R\+Jrt9j\+Ngej0\+F8\+Svvfy\+Jj\+Y/edit?usp=sharing}

\subsection*{Dependencies}


\begin{DoxyEnumerate}
\item C++ 11/14/17
\item gnuplot ~\newline
Install gnuplot 
\begin{DoxyCode}
1 sudo apt-get install gnuplot
\end{DoxyCode}

\item boost ~\newline
Install boost 
\begin{DoxyCode}
1 sudo apt-get install libboost-all-dev
\end{DoxyCode}

\end{DoxyEnumerate}

\#\# Build 
\begin{DoxyCode}
1 git clone --recursive https://github.com/RajPShinde/Robot\_Controller\_Module.git
2 cd <path to repository>
3 mkdir build
4 cd build
5 cmake ..
6 make
\end{DoxyCode}
 \subsection*{Run}

\#\#\#\# Run Program 
\begin{DoxyCode}
1 ./app/shell-app
\end{DoxyCode}
 \#\#\#\# Run Test 
\begin{DoxyCode}
1 ./test/cpp-test
\end{DoxyCode}
 \subsection*{Demo}

Run the program. Once the velocity and heading converge to the target then graphs will be displayed as below. Also, converged values will be shown in the terminal.

\subparagraph*{Heading Convergence}

 

\subparagraph*{Velocity Convergence}

 

\subparagraph*{Terminal Output}

 

\subsection*{Bugs}

None

\subsection*{References}


\begin{DoxyItemize}
\item Ackermann Steering-\/ \href{https://www.sciencedirect.com/topics/engineering/ackermann}{\tt https\+://www.\+sciencedirect.\+com/topics/engineering/ackermann}
\item P\+ID Controller-\/ \href{https://en.wikipedia.org/wiki/PID_controller}{\tt https\+://en.\+wikipedia.\+org/wiki/\+P\+I\+D\+\_\+controller}
\item gnuplot-\/ \href{http://stahlke.org/dan/gnuplot-iostream/}{\tt http\+://stahlke.\+org/dan/gnuplot-\/iostream/} 
\end{DoxyItemize}